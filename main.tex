\documentclass{article}
\usepackage[table,xcdraw]{xcolor}
\usepackage[dutch]{babel}
\usepackage{float}
\usepackage{xcolor}
\usepackage{graphicx}
\usepackage{amsmath}
\usepackage{verbatim}
\usepackage{listings}
\usepackage{parskip}
\usepackage{csquotes}
\usepackage{biblatex}
\usepackage[colorlinks=true]{hyperref}
\usepackage[dutch]{cleveref}

% Hieronder een alternatief om de indentatie bij nieuwe paragrafen manueel uit te zetten
% \setlength\parindent{0pt} 
\addbibresource{references.bib}

\title{LaTeX Workshop}
\author{Allyson Robert}
\date{\today}

\begin{document}

\maketitle
\tableofcontents

\section{Introduction}
\label{sec:intro}
Hier is een willekeurige inleidingende paragraaf. 
Als jullie zover geraakt zijn dan kunnen we verder met het volgend onderdeel.
Het is altijd handig om elke zin op een nieuwe regel te plaatsen, zo zullen in de toekomst errors gemakkelijker getraceerd worden.

\section{De wetten van Newton}
Voor zij die het niet meer weten, de wetten van Newton kunnen als volgt genoteerd worden:

\begin{align}
    \sum_{i=0}^{N} \vec{F} = 0 &\Rightarrow \vec{v} = \text{cte}\\
    %
    \vec{a} &= \frac{\sum \vec{F}}{m} \\
    %
    \vec{F}_{A\to B} &= -\vec{F}_{B \to A}
\end{align}

waarbij $\vec{F}_{A \to B}$ een kracht voorstelt van lichaam $A$ op lichaam $B$.
Handige tips voor wiskundige uitdrukkingen kun je vinden op de website van Overleaf \cite{OverleafMath}.

\section{Figuren}
    \begin{figure}[H]
    \centering
    \includegraphics[width=\textwidth]{img/pikachu_transparent.png}
    \caption{Fat Pikachu is best Pikachu}
    \label{fig:pikachu}
\end{figure}

\section{Tabellen}

\begin{table}[H]
\centering
\caption{Plotting tool dependencies}
\label{tab:dependencies}
\begin{tabular}{l|l|l}
\rowcolor[HTML]{EFEFEF} 
Data & GUI & Built-in \\ \hline\hline
Plotly v5.11.0 & PyQt5 v5.15.4 & datetime \\ 
Numpy v1.24.0 & natsort v8.2.0 & json \\ 
Pandas v1.5.2 &  & csv \\ 
 &  & sys \\ 
 &  & os \\ 
\end{tabular}
\end{table}

Ook voor tabellen en figuren heeft Overleaf nuttige informatie\cite{OverleafFigures}, voor minipages kan je kijken op StackExchange\cite{StackExchangeMinipage}.

\section{Code snippets}
Hieronder heb ik de package \textit{listings} gebruikt die keywords bevat voor veel bekende talen zoals Python en C++.
Let echter op daar listings beschouwd worden als floating.

\begin{lstlisting}[language=C++]
class Particle {
public:
    // Constructor
    Particle(double mass, double charge) 
        : mass(mass), charge(charge) {}

    // Getters
    double getMass() const { return mass; }
    double getCharge() const { return charge; }

    // Setters
    void setMass(double m) { mass = m; }
    void setCharge(double c) { charge = c; }

private:
    double mass;
    double charge;
};
\end{lstlisting}
\begin{lstlisting}[language=Python,label=listing:python,caption=Python example with caption and label]    
import webbrowser

def open_youtube_video():
    url = "https://www.youtube.com/watch?v=dQw4w9WgXcQ"
    webbrowser.open(url)
open_youtube_video()
\end{lstlisting}

Dit bekom je dankzij de volgende structuur (hier letterlijk geplaatst dankzij de package \textit{verbatim}):
\begin{verbatim}
    \begin{lstlisting}[language=Python,label=listing:python,
    caption=Python example with caption and label]    
    import webbrowser
    
    def open_youtube_video():
        url = "https://www.youtube.com/watch?v=dQw4w9WgXcQ"
        webbrowser.open(url)
    open_youtube_video()
\end{lstlisting}
\end{verbatim}

\section{Referenties}
Je kan refereren naar labels in je tekst/figure/tabellen etc. door bijvoorbeeld \verb|\ref{fig:pikachu}| te gebruiken.
Zo verwijst \LaTeX ~automatisch naar het juiste nummer van de figuur: \ref{fig:pikachu}.
Beter is om de cleveref package en het commando \verb|\cref{tab:dependencies}| te gebruiken zodat LaTeX ook zelf refereert naar het type object.
Zo weet latex zelf dat ik refereer naar \cref{fig:pikachu} en \cref{tab:dependencies} of zelfs \cref{sec:intro}.
Dit zal afhangen van de taalinstellingen in je preamble, gebruik hiervoor de juiste instellingen van de \textit{babel} package.

Het is ook mogelijk om LaTeX je bibliografie te laten beheren.
Gebruik daarvoor de \textit{csquotes} en \textit{biblatex}.
Je dient in de preamble ook te wijzen naar een \textit{references.bib} bestand met \verb|addbibresource| en \verb|\printbibliography| gebruiken waar je de lijst van referenties wilt toevoegen.
Zo kan ik enkele handige pagina's van Overleaf citeren \cite{Overleaf30min}.
Het document heb ik aangevuld met meerdere dergelijke referenties.

\printbibliography




\end{document}
